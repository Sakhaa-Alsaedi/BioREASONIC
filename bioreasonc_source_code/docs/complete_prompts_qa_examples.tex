%% BioREASONIC-Bench: Complete Prompts, Questions, and Answer Examples
%% Shows the full flow: Prompt → Question → Expert Answer

\section{Complete Prompts with Question-Answer Examples}

This document provides the complete generation flow for each taxonomy:
\textbf{System Prompt} → \textbf{Task Prompt} → \textbf{Generated Question} → \textbf{Expert Answer}

%% ============================================================================
\subsection{System Prompt (Used for All Taxonomies)}
%% ============================================================================

\begin{tcolorbox}[colback=gray!10,colframe=gray!50!black,title=EXPERT\_SYSTEM\_PROMPT]
\begin{verbatim}
You are an expert biomedical geneticist with deep knowledge of:
- GWAS methodology and interpretation
- Mendelian Randomization and causal inference
- Molecular biology and disease pathophysiology
- Clinical genetics and precision medicine

When answering questions about gene-disease associations:

1. EXPLAIN THE BIOLOGY: Don't just cite numbers. Explain what the gene
   encodes, its molecular function, and how dysfunction leads to disease.

2. USE NATURAL LANGUAGE: Write as an expert explaining to a colleague,
   not as a template being filled in. Avoid "Step 1:", "Step 2:" formats.

3. PROVIDE CONTEXT: Put statistics in perspective. Compare effect sizes
   to typical GWAS findings. Explain what odds ratios mean clinically.

4. CONNECT EVIDENCE TO MECHANISM: Link statistical findings (GWAS, MR)
   to biological pathways that explain WHY the association exists.

5. ACKNOWLEDGE NUANCE: Note limitations, alternative explanations, and
   areas of uncertainty. Use appropriate hedging language.

6. BE OBJECTIVE: State conclusions based on evidence without personal
   opinion framing like "I think" or "My interpretation."
\end{verbatim}
\end{tcolorbox}

%% ============================================================================
%% STRUCTURE (S) TAXONOMY
%% ============================================================================
\subsection{S (Structure-Aware) Taxonomy}

%% --- EXAMPLE S-1: SNP-Gene Association ---
\subsubsection{Example S-1: SNP-Gene Association Support}

\begin{tcolorbox}[colback=blue!5,colframe=blue!40!black,title=S\_EXPERT\_PROMPT]
\begin{verbatim}
You are explaining the genetic architecture of a gene-disease association.

Gene: IL21R
Disease: Asthma
SNP count: 136
Unique independent signals: 20
Evidence level: very_strong
Gene function: the receptor for interleukin-21, a cytokine that
              orchestrates immune responses including T cell activation,
              B cell differentiation, and IgE class switching

Question: Is the IL21R-Asthma association supported by multiple
independent genetic variants?

Instructions:
1. Answer yes/no with the SNP statistics
2. Explain what "independent signals" means biologically (LD, fine-mapping)
3. Describe what IL21R encodes and its biological function
4. Explain HOW IL21R dysfunction contributes to Asthma pathophysiology
5. Interpret what multiple independent signals suggest about the
   gene's causal role
6. Write in natural expert prose, not numbered steps
\end{verbatim}
\end{tcolorbox}

\textbf{Generated Question:}
\begin{tcolorbox}[colback=white,colframe=blue!60!black]
Is the IL21R-Asthma association supported by multiple independent genetic variants?
\end{tcolorbox}

\textbf{Expert Answer:}
\begin{tcolorbox}[colback=blue!5,colframe=blue!60!black,title=Expert Answer]
Yes, and the genetic architecture strongly supports IL21R as a true asthma susceptibility gene.

The association is backed by 136 SNPs, of which 20 represent independent signals after accounting for linkage disequilibrium. This distinction matters: a single causal variant can produce many associated SNPs through LD, but 20 \textit{independent} signals suggest multiple functional variants within or near IL21R affecting asthma risk.

From a biological standpoint, this makes sense. IL21R encodes the receptor for interleukin-21, a cytokine that orchestrates immune responses. In the context of asthma:

\begin{itemize}
    \item IL-21 signaling drives B cell differentiation and IgE class switching—the antibody responsible for allergic reactions
    \item It modulates the Th1/Th2 balance, and asthma is fundamentally a Th2-driven disease
    \item IL-21 also affects regulatory T cells that normally suppress excessive immune responses
\end{itemize}

The presence of multiple independent genetic signals suggests that several different variants—perhaps affecting expression levels, splicing, or protein function—each contribute to asthma susceptibility through this immunological pathway. This is consistent with IL21R being a genuine causal gene rather than simply tagging a nearby causal locus.
\end{tcolorbox}

%% --- EXAMPLE S-2: Graph Traversal ---
\subsubsection{Example S-2: Biological Pathway Traversal}

\begin{tcolorbox}[colback=blue!5,colframe=blue!40!black,title=S\_GRAPH\_TRAVERSAL\_PROMPT]
\begin{verbatim}
You are explaining the biological pathway from a genetic variant to disease.

Variant: rs1799884
Gene: GCK
Disease: Type 2 Diabetes
Gene function: glucokinase, a hexokinase that acts as the pancreatic
              glucose sensor by catalyzing the rate-limiting step of
              glucose phosphorylation in beta cells

Question: Explain the biological path from variant rs1799884 to
Type 2 Diabetes.

Instructions:
1. Describe where the variant is located (promoter, coding, intronic)
2. Explain what the gene encodes and its molecular function
3. Describe the step-by-step biological pathway:
   Variant → Gene → Protein → Function → Disease
4. Explain the mechanism at each step
5. Note if there is monogenic disease evidence (e.g., MODY)
6. Write as a cohesive biological narrative
\end{verbatim}
\end{tcolorbox}

\textbf{Generated Question:}
\begin{tcolorbox}[colback=white,colframe=blue!60!black]
Explain the biological path from variant rs1799884 to Type 2 Diabetes.
\end{tcolorbox}

\textbf{Expert Answer:}
\begin{tcolorbox}[colback=blue!5,colframe=blue!60!black,title=Expert Answer]
The rs1799884 variant sits in the promoter region of GCK, which encodes glucokinase—often called the pancreatic ``glucose sensor.''

Here's why this matters for diabetes:

Glucokinase is a hexokinase that phosphorylates glucose in pancreatic beta cells, but unlike other hexokinases, it has a relatively high Km for glucose ($\sim$8 mM). This means it only becomes active when blood glucose rises above fasting levels. This property makes it the rate-limiting step that couples blood glucose concentration to insulin secretion.

The rs1799884 variant reduces GCK promoter activity by approximately 15-20\%. The consequence is subtle but significant:

\begin{enumerate}
    \item Beta cells express less glucokinase
    \item Higher glucose concentrations are needed to trigger insulin release
    \item Fasting glucose creeps upward (the hallmark of early T2D)
    \item Over years, chronic mild hyperglycemia contributes to beta cell exhaustion
\end{enumerate}

Notably, complete GCK loss-of-function causes MODY2 (Maturity-Onset Diabetes of the Young, type 2), a monogenic form of diabetes. The common rs1799884 variant represents the milder end of this spectrum—partial reduction in glucose sensing that, combined with other genetic and lifestyle factors, contributes to polygenic T2D risk.

This variant is a textbook example of how GWAS findings connect to established diabetes biology through a well-characterized metabolic pathway.
\end{tcolorbox}

%% ============================================================================
%% CAUSAL (C) TAXONOMY
%% ============================================================================
\subsection{C (Causal-Aware) Taxonomy}

%% --- EXAMPLE C-1: Association vs Causation ---
\subsubsection{Example C-1: Association vs. Causation}

\begin{tcolorbox}[colback=green!5,colframe=green!40!black,title=C\_EXPERT\_PROMPT]
\begin{verbatim}
You are explaining causal inference in genetic epidemiology.

Gene: TCF7L2
Disease: Type 2 Diabetes
GWAS OR: 1.37
GWAS p-value: 1e-200
MR score: 0.85
Has MR support: True
Gene function: a transcription factor in the Wnt signaling pathway that
              regulates beta cell proliferation, survival, and incretin
              (GLP-1) responses

Question: Based on GWAS evidence alone, can we conclude that TCF7L2
causes Type 2 Diabetes?

Instructions:
1. Clearly state NO - GWAS shows association, not causation
2. Explain WHY association ≠ causation (confounding, LD, reverse causation)
3. Describe what TCF7L2 actually does biologically
4. Explain the biological pathway: TCF7L2 → protein → function → T2D
5. Describe what additional evidence (MR, functional studies) would be
   needed and why
6. If MR evidence exists, explain what it adds to the causal argument
7. Provide the CORRECT language: "associated with" not "causes"
8. Write as a natural explanation, not mechanical steps
\end{verbatim}
\end{tcolorbox}

\textbf{Generated Question:}
\begin{tcolorbox}[colback=white,colframe=green!60!black]
Based on GWAS evidence alone, can we conclude that TCF7L2 causes Type 2 Diabetes?
\end{tcolorbox}

\textbf{Expert Answer:}
\begin{tcolorbox}[colback=green!5,colframe=green!60!black,title=Expert Answer]
No—and this distinction is crucial for interpreting genetic studies correctly.

GWAS tells us that people carrying certain TCF7L2 variants (particularly rs7903146) have higher T2D rates. The association is robust: OR of 1.37, replicated across dozens of studies and multiple populations. But association is not causation.

Consider the alternatives:

\textbf{Confounding:} Perhaps TCF7L2 variants are more common in populations with dietary patterns that independently increase T2D risk. The association would be real but not causal.

\textbf{Linkage disequilibrium:} rs7903146 might simply tag the true causal variant nearby. TCF7L2 spans over 200kb; the causal variant could be in a regulatory element affecting a different gene entirely.

\textbf{Reverse causation:} Less likely for germline variants, but metabolic changes in pre-diabetes could theoretically affect TCF7L2 regulation.

TCF7L2 encodes a transcription factor in the Wnt signaling pathway that regulates beta cell proliferation, survival, and incretin (GLP-1) responses. The biological pathway to T2D involves:

\begin{itemize}
    \item TCF7L2 controls genes essential for beta cell function
    \item Risk variants reduce TCF7L2 activity in islets
    \item Impaired GLP-1 signaling reduces glucose-stimulated insulin secretion
    \item Progressive beta cell dysfunction leads to hyperglycemia
\end{itemize}

TCF7L2 is actually well-supported causally—MR score of 0.85 supports a causal relationship, and functional studies show the risk allele reduces insulin secretion. But here's the key point: we need that \textit{additional} evidence beyond GWAS.

The correct statement is: ``TCF7L2 variants are associated with increased T2D risk.'' Claiming causation requires MR, functional studies, or ideally both.
\end{tcolorbox}

%% --- EXAMPLE C-2: MR Evidence ---
\subsubsection{Example C-2: Mendelian Randomization Evidence}

\begin{tcolorbox}[colback=green!5,colframe=green!40!black,title=C\_MR\_EXPERT\_PROMPT]
\begin{verbatim}
You are explaining Mendelian Randomization evidence for causation.

Gene: PCSK9
Disease: Coronary Artery Disease
MR score: 0.95
Gene function: proprotein convertase subtilisin/kexin type 9, which binds
              to LDL receptors on hepatocytes and promotes their
              degradation, thereby regulating circulating LDL-cholesterol
Biological pathway: PCSK9 → LDL receptor degradation → LDL-C levels → CAD

Question: Does Mendelian Randomization evidence support a causal role
for PCSK9 in coronary artery disease?

Instructions:
1. Explain MR methodology (genetic variants as instrumental variables)
2. Interpret the MR score in context
3. Describe the biological mechanism that EXPLAINS why the causal
   relationship exists
4. Note if there is therapeutic validation (drugs targeting this pathway)
5. Discuss MR assumptions and potential violations
6. Write as natural expert explanation
\end{verbatim}
\end{tcolorbox}

\textbf{Generated Question:}
\begin{tcolorbox}[colback=white,colframe=green!60!black]
Does Mendelian Randomization evidence support a causal role for PCSK9 in coronary artery disease?
\end{tcolorbox}

\textbf{Expert Answer:}
\begin{tcolorbox}[colback=green!5,colframe=green!60!black,title=Expert Answer]
Yes—PCSK9 is perhaps the clearest example of genetics validating a drug target through MR.

The logic of MR is elegant: genetic variants are assigned at conception, before disease develops, so they can't be affected by reverse causation. If PCSK9 variants that lower LDL-cholesterol also lower CAD risk proportionally, it suggests the LDL-lowering \textit{itself} (not some confounder) protects against CAD.

The MR score of 0.95 indicates very strong causal support.

The evidence is compelling: Loss-of-function PCSK9 variants (like R46L) reduce LDL-C by about 15\% and CAD risk by roughly 25\%. Crucially, this reduction matches what we'd predict from statin trials—the ``LDL slope'' is consistent whether you lower LDL with statins, PCSK9 variants, or PCSK9 inhibitors.

The biology explains why: PCSK9 normally binds to LDL receptors on liver cells and promotes their degradation. Less PCSK9 activity means more LDL receptors, more LDL clearance, lower circulating LDL, less atherosclerosis:

\begin{verbatim}
PCSK9 ↓ → LDL receptors ↑ → LDL clearance ↑ → LDL-C ↓ → Atherosclerosis ↓ → CAD ↓
\end{verbatim}

What makes PCSK9 special is the therapeutic validation. MR predicted that PCSK9 inhibition would reduce cardiovascular events. The FOURIER and ODYSSEY trials confirmed exactly this—evolocumab and alirocumab reduced heart attacks and strokes by 15-20\%.

This represents causal inference at its best: genetic epidemiology generating a hypothesis, mechanistic biology explaining it, and randomized trials confirming it. PCSK9's causal role in CAD is now considered established.
\end{tcolorbox}

%% ============================================================================
%% RISK (R) TAXONOMY
%% ============================================================================
\subsection{R (Risk-Aware) Taxonomy}

%% --- EXAMPLE R-1: Risk Factor Assessment ---
\subsubsection{Example R-1: Risk Factor Assessment}

\begin{tcolorbox}[colback=orange!5,colframe=orange!40!black,title=R\_EXPERT\_PROMPT]
\begin{verbatim}
You are explaining genetic risk factor assessment.

Gene: APOE
Disease: Alzheimer's Disease
Risk score: 0.95
Evidence level: very_strong
SNP count: 2 (ε2/ε3/ε4 haplotypes)
OR: 3.2 (heterozygous ε4), 14.9 (homozygous ε4)
MR score: 0.90
Gene function: apolipoprotein E, the brain's primary cholesterol
              transporter involved in lipid metabolism, amyloid clearance,
              and neuronal maintenance

Question: Should APOE be considered a genetic risk factor for
Alzheimer's disease?

Instructions:
1. State the conclusion (yes/no) with evidence strength
2. Provide the statistical evidence (OR, SNPs, risk score)
3. PUT NUMBERS IN CONTEXT: Compare to typical GWAS effect sizes
4. Explain WHAT APOE encodes and its molecular function
5. Describe the BIOLOGICAL MECHANISM linking APOE to AD
6. Explain what the effect size means for individual risk
7. Note clinical implications if relevant
8. Write as natural expert explanation
\end{verbatim}
\end{tcolorbox}

\textbf{Generated Question:}
\begin{tcolorbox}[colback=white,colframe=orange!60!black]
Should APOE be considered a genetic risk factor for Alzheimer's disease?
\end{tcolorbox}

\textbf{Expert Answer:}
\begin{tcolorbox}[colback=orange!5,colframe=orange!60!black,title=Expert Answer]
APOE isn't just a risk factor for Alzheimer's—it's the most important genetic risk factor by a considerable margin.

The numbers are striking. Carrying one APOE $\varepsilon$4 allele roughly triples your Alzheimer's risk (OR $\approx$ 3.2); carrying two copies increases it roughly 12-15 fold. No other common variant comes close. For context, most GWAS hits for complex diseases have odds ratios between 1.1 and 1.3.

Why such a large effect? APOE (apolipoprotein E) is the brain's primary cholesterol transporter, and it turns out to be intimately involved in multiple Alzheimer's pathways:

\textbf{Amyloid clearance:} The $\varepsilon$4 isoform is less efficient at clearing amyloid-$\beta$ from brain tissue. A$\beta$ accumulates, forms plaques, and triggers the cascade of neurodegeneration. PET imaging shows $\varepsilon$4 carriers accumulate amyloid earlier and faster.

\textbf{Tau pathology:} Independent of amyloid, APOE4 promotes tau phosphorylation and the spread of tau tangles through the brain. This may explain why some $\varepsilon$4 carriers develop Alzheimer's-type dementia even without heavy amyloid burden.

\textbf{Neuroinflammation:} APOE4 shifts microglia toward a pro-inflammatory state, impairing their normal housekeeping functions and potentially accelerating synaptic loss.

\textbf{Vascular contributions:} APOE4 is associated with blood-brain barrier dysfunction, allowing peripheral factors to enter the brain.

Clinically, APOE genotyping is increasingly used for risk stratification, especially in research settings. However, it's important to note that APOE4 is neither necessary nor sufficient for Alzheimer's—many carriers never develop dementia, and many patients lack the $\varepsilon$4 allele. It's a powerful risk modifier operating within a multifactorial disease.
\end{tcolorbox}

%% --- EXAMPLE R-2: OR Interpretation ---
\subsubsection{Example R-2: Odds Ratio Interpretation}

\begin{tcolorbox}[colback=orange!5,colframe=orange!40!black,title=R\_OR\_EXPERT\_PROMPT]
\begin{verbatim}
You are explaining odds ratio interpretation in genetic epidemiology.

Gene: GCK
Disease: Type 2 Diabetes
OR: 1.45
Gene function: glucokinase, the pancreatic glucose sensor
Baseline population risk: ~10% lifetime risk

Question: What does an odds ratio of 1.45 for GCK variants in
Type 2 Diabetes mean?

Instructions:
1. Explain what OR means mathematically (odds increase per risk allele)
2. PUT IN CONTEXT: Compare to typical GWAS effect sizes
3. Explain WHY the effect size is what it is based on biology
4. Calculate what this means for individual risk
5. Note the difference between relative and absolute risk
6. Mention lifestyle factors that may have larger absolute effects
7. Write as natural explanation
\end{verbatim}
\end{tcolorbox}

\textbf{Generated Question:}
\begin{tcolorbox}[colback=white,colframe=orange!60!black]
What does an odds ratio of 1.45 for GCK variants in Type 2 Diabetes mean?
\end{tcolorbox}

\textbf{Expert Answer:}
\begin{tcolorbox}[colback=orange!5,colframe=orange!60!black,title=Expert Answer]
An OR of 1.45 means that each copy of the GCK risk allele increases your odds of developing T2D by 45\%. But let me put that in perspective.

In the world of complex disease genetics, 1.45 is actually a substantial effect. Most GWAS variants for T2D, heart disease, and other common conditions have ORs between 1.05 and 1.20. GCK sits among the stronger genetic effects, alongside variants in TCF7L2 and FTO.

Why is GCK's effect relatively large? Because it's not just statistically associated with diabetes—it sits directly in the causal pathway. GCK encodes glucokinase, the enzyme that determines when beta cells release insulin in response to glucose. Reduce GCK activity, and you shift the glucose threshold for insulin secretion upward. This is exactly what happens in MODY2, where complete loss-of-function mutations cause a monogenic form of diabetes.

The common rs1799884 variant is milder—it reduces GCK expression perhaps 15-20\%—but the effect is in the same direction along the same pathway. This biological directness explains the relatively large OR.

Now, what does 1.45 mean for an individual?

If your baseline T2D risk is 10\%, carrying the risk allele increases it to about 13\%. Meaningful, but not deterministic. If your baseline risk is 40\% (due to family history, obesity, etc.), it pushes you toward 50\%.

The key insight is that OR describes \textit{relative} risk. A 45\% increase is dramatic if your baseline is high, but barely noticeable if it's low. For most people, lifestyle factors (weight, diet, exercise) will have larger absolute effects on their T2D risk than any single genetic variant—even one with a relatively large OR like GCK.
\end{tcolorbox}

%% ============================================================================
%% MECHANISM (M) TAXONOMY
%% ============================================================================
\subsection{M (Mechanism-Aware) Taxonomy}

%% --- EXAMPLE M-1: Mechanistic Pathway ---
\subsubsection{Example M-1: Mechanistic Pathway Analysis}

\begin{tcolorbox}[colback=purple!5,colframe=purple!40!black,title=M\_EXPERT\_PROMPT]
\begin{verbatim}
You are explaining the biological mechanism connecting a gene to disease.

Gene: HLA-DRB1
Disease: Rheumatoid Arthritis
GO functional score: 0.85
Has pathway support: True
Gene function: part of the MHC class II complex that presents peptide
              antigens to CD4+ T cells, initiating adaptive immune responses
GO terms: GO:0002504 (antigen presentation via MHC-II),
          GO:0019886 (antigen processing), GO:0050852 (T cell signaling)
PPI partners: CD4, CIITA, CD74, PTPN22

Question: What is the mechanistic basis for HLA-DRB1's role in
Rheumatoid Arthritis?

Instructions:
1. Start with WHAT the gene encodes (protein name, function)
2. Describe relevant GO terms and what they mean biologically
3. Explain the PPI network context (what proteins it interacts with)
4. MOST IMPORTANT: Describe the step-by-step biological pathway:
   Gene → Protein → Molecular function → Cellular effect →
   Tissue/organ impact → Disease phenotype
5. Connect pathway to disease pathophysiology
6. Note if the mechanism is well-established or hypothetical
7. Mention clinical/therapeutic implications if relevant
8. Write as cohesive biological narrative, not bullet points
\end{verbatim}
\end{tcolorbox}

\textbf{Generated Question:}
\begin{tcolorbox}[colback=white,colframe=purple!60!black]
What is the mechanistic basis for HLA-DRB1's role in Rheumatoid Arthritis?
\end{tcolorbox}

\textbf{Expert Answer:}
\begin{tcolorbox}[colback=purple!5,colframe=purple!60!black,title=Expert Answer]
HLA-DRB1 is central to rheumatoid arthritis pathogenesis, and we actually understand the mechanism in considerable detail—a rarity in complex disease genetics.

HLA-DRB1 encodes part of the MHC class II complex, whose job is presenting peptide fragments to CD4+ T cells. The key insight came from noticing that RA-associated HLA-DRB1 alleles (*04:01, *04:04, *01:01) share a common amino acid sequence at positions 70-74: the ``shared epitope.''

Here's how this drives RA:

In RA, the immune system attacks the joints. But why? The current model involves citrullination—a post-translational modification where arginine residues are converted to citrulline by PAD enzymes. This happens during inflammation and cell death.

The shared epitope preferentially binds citrullinated peptides. When citrullinated self-proteins (like fibrinogen, vimentin, or collagen) are presented by HLA-DR molecules carrying the shared epitope, they can activate T cells that escaped normal tolerance mechanisms.

These autoreactive T cells then:
\begin{enumerate}
    \item Help B cells produce anti-citrullinated protein antibodies (ACPAs)
    \item ACPAs form immune complexes in the synovium
    \item Immune complexes activate complement and recruit macrophages
    \item Chronic inflammation leads to pannus formation—invasive synovial tissue that destroys cartilage and bone
\end{enumerate}

The PPI network reflects this biology. HLA-DRB1 interacts with CD4 (the T cell co-receptor), CIITA (the master regulator of MHC II expression), and CD74 (involved in peptide loading). Remarkably, another major RA risk gene, PTPN22, also modulates T cell activation thresholds—the genetic architecture converges on the same immunological pathway.

This mechanistic understanding has clinical implications: it explains why RA is strongly associated with ACPA positivity, and it suggests that tolerance-inducing therapies targeting citrullinated peptides might be therapeutic.
\end{tcolorbox}

%% --- EXAMPLE M-2: Functional Connection ---
\subsubsection{Example M-2: Functional Connection Assessment}

\begin{tcolorbox}[colback=purple!5,colframe=purple!40!black,title=M\_FUNCTIONAL\_CONNECTION\_PROMPT]
\begin{verbatim}
You are assessing the functional connection between a gene and disease.

Gene: SLC30A8
Disease: Type 2 Diabetes
GO score: 0.78
Gene function: ZnT8, a zinc transporter specifically expressed in
              pancreatic beta cells that moves zinc into insulin
              secretory granules for proper insulin crystallization

Question: Is SLC30A8 functionally connected to Type 2 Diabetes based
on PPI networks and GO terms?

Instructions:
1. State yes/no with the GO score
2. Explain what the gene encodes
3. Describe the relevant GO biological processes
4. Explain HOW this function relates to the disease mechanism
5. Note any interesting findings (e.g., protective vs risk, paradoxes)
6. Mention therapeutic implications if relevant
\end{verbatim}
\end{tcolorbox}

\textbf{Generated Question:}
\begin{tcolorbox}[colback=white,colframe=purple!60!black]
Is SLC30A8 functionally connected to Type 2 Diabetes based on PPI networks and GO terms?
\end{tcolorbox}

\textbf{Expert Answer:}
\begin{tcolorbox}[colback=purple!5,colframe=purple!60!black,title=Expert Answer]
Yes, and SLC30A8 is one of the more interesting diabetes genes because of what loss-of-function variants tell us.

SLC30A8 encodes ZnT8, a zinc transporter found almost exclusively in pancreatic beta cells. Its job is moving zinc into insulin secretory granules, where zinc is essential for insulin crystallization and storage. Insulin is stored as zinc-insulin hexamers; without adequate zinc, insulin packaging goes awry.

The relevant GO terms include:
\begin{itemize}
    \item GO:0006882 — Cellular zinc ion homeostasis
    \item GO:0030073 — Insulin secretion
    \item GO:0005385 — Zinc ion transmembrane transporter activity
\end{itemize}

The functional connection to diabetes seemed straightforward: disrupt ZnT8, impair insulin secretion, increase diabetes risk. GWAS confirmed an association. Case closed?

Not quite. When researchers looked at rare loss-of-function SLC30A8 variants, they found something surprising: these variants actually \textit{protect} against T2D, with roughly 65\% reduced risk. This was the opposite of what the common GWAS variant suggested.

This apparent paradox likely reflects different mechanisms:
\begin{itemize}
    \item The common risk variant may affect ZnT8 regulation or localization
    \item Complete loss-of-function triggers compensatory mechanisms
    \item Reduced zinc transport might actually improve insulin processing in some contexts
\end{itemize}

The clinical implication is significant: ZnT8 inhibition could be a therapeutic strategy for T2D. Several companies are exploring this, guided by the genetic evidence that people with reduced ZnT8 function are healthier, not sicker.

This exemplifies how deep functional understanding of GWAS hits can reveal unexpected biology and, occasionally, counterintuitive drug targets.
\end{tcolorbox}

%% ============================================================================
\subsection{Summary: Prompt → Question → Answer Flow}
%% ============================================================================

\begin{table}[h]
\centering
\caption{Complete Generation Flow Summary}
\begin{tabular}{p{1.5cm}p{4cm}p{4cm}p{4cm}}
\toprule
\textbf{Tax.} & \textbf{Prompt Focus} & \textbf{Question Type} & \textbf{Answer Includes} \\
\midrule
S & Genetic architecture, LD, independent signals & SNP support for association & SNP stats + gene function + biological pathway \\
\midrule
C & Causal inference, confounding, MR methodology & Association vs. causation & Why GWAS $\neq$ causation + mechanism + correct language \\
\midrule
R & Risk assessment, OR interpretation, clinical meaning & Risk factor evaluation & OR context + biological mechanism + individual risk meaning \\
\midrule
M & GO terms, PPI networks, pathway biology & Mechanistic basis & Step-by-step pathway + PPI context + clinical implications \\
\bottomrule
\end{tabular}
\end{table}

%% ============================================================================
%% END OF DOCUMENT
%% ============================================================================
