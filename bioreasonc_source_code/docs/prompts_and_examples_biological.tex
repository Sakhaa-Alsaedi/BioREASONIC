%% BioREASONIC-Bench: Prompts and Question Examples with Biological Context
%% This document shows IDEAL examples with proper biological explanations

\section{Question Generation: Ideal Examples with Biological Context}

This document provides exemplary question-answer pairs that demonstrate proper biological reasoning. Each answer explains the \textbf{biological mechanism} connecting the gene to the disease, not just statistical metrics.

%% ============================================================================
\subsection{Structure-Aware (S) Taxonomy}
%% ============================================================================

\subsubsection{Ideal Example S-1: SNP-Gene-Disease Mapping}

\begin{tcolorbox}[colback=blue!5,colframe=blue!40!black,title=S: Structure-Aware Question]
\textbf{Question:} Is the IL21R-Asthma association supported by multiple independent genetic variants? Explain the biological basis.

\textbf{Ideal Answer:}
Yes. The IL21R-Asthma association is supported by 136 SNPs (20 unique independent signals).

\textbf{Biological Explanation:}
IL21R (Interleukin-21 Receptor) is a cytokine receptor expressed on T cells, B cells, and NK cells. It plays a critical role in:
\begin{itemize}
    \item \textbf{Th2 immune response regulation}: IL-21 signaling modulates the balance between Th1 and Th2 responses
    \item \textbf{IgE production}: IL21R activation influences B cell class switching to IgE, which drives allergic inflammation
    \item \textbf{Airway inflammation}: Dysregulated IL-21 signaling contributes to eosinophilic airway inflammation characteristic of asthma
\end{itemize}

The multiple independent genetic signals (20 unique variants) suggest that IL21R harbors multiple functional variants affecting immune regulation, strengthening its role as a true asthma susceptibility gene rather than a tagging artifact.
\end{tcolorbox}

\subsubsection{Ideal Example S-2: Multi-hop Graph Traversal}

\begin{tcolorbox}[colback=blue!5,colframe=blue!40!black,title=S: Graph Traversal with Biological Path]
\textbf{Question:} Trace the biological path from variant rs1799884 to Type 2 Diabetes.

\textbf{Ideal Answer:}

\textbf{Graph Traversal:}
\begin{verbatim}
rs1799884 (Variant)
    |
    | [located in promoter region]
    v
GCK (Glucokinase gene)
    |
    | [encodes]
    v
Glucokinase enzyme (hexokinase IV)
    |
    | [functions as glucose sensor in]
    v
Pancreatic beta cells
    |
    | [regulates]
    v
Insulin secretion threshold
    |
    | [when impaired, leads to]
    v
Elevated fasting glucose → Type 2 Diabetes
\end{verbatim}

\textbf{Biological Mechanism:}
GCK acts as the pancreatic glucose sensor by catalyzing the rate-limiting step of glucose phosphorylation (glucose $\rightarrow$ glucose-6-phosphate). The variant rs1799884 in the GCK promoter reduces transcriptional activity, leading to:
\begin{enumerate}
    \item Decreased glucokinase expression in beta cells
    \item Higher glucose threshold required to trigger insulin release
    \item Chronic mild hyperglycemia (MODY2 phenotype in severe cases)
    \item Progressive beta cell dysfunction contributing to T2D
\end{enumerate}
\end{tcolorbox}

%% ============================================================================
\subsection{Causal-Aware (C) Taxonomy}
%% ============================================================================

\subsubsection{Ideal Example C-1: Association vs. Causation}

\begin{tcolorbox}[colback=green!5,colframe=green!40!black,title=C: Causal Reasoning with Biological Context]
\textbf{Question:} Based on GWAS evidence alone, can we conclude that TCF7L2 causes Type 2 Diabetes?

\textbf{Ideal Answer:}
\textbf{No, we cannot conclude causation from GWAS alone.}

\textbf{Step 1 - What GWAS Shows:}
GWAS identifies that TCF7L2 variants (lead SNP rs7903146, OR=1.37) are statistically \textit{associated} with T2D risk.

\textbf{Step 2 - Why Association $\neq$ Causation:}
\begin{itemize}
    \item \textbf{Confounding}: Population stratification or correlated environmental factors could explain the association
    \item \textbf{Linkage Disequilibrium}: rs7903146 may tag another causal variant nearby
    \item \textbf{Reverse causation}: Metabolic changes in T2D could theoretically affect TCF7L2 regulation (though unlikely for germline variants)
\end{itemize}

\textbf{Step 3 - Biological Plausibility (supportive but not conclusive):}
TCF7L2 encodes a transcription factor in the Wnt signaling pathway that regulates:
\begin{itemize}
    \item Beta cell proliferation and survival
    \item Incretin hormone (GLP-1) signaling
    \item Hepatic gluconeogenesis
\end{itemize}
While biologically plausible, plausibility alone does not prove causation.

\textbf{Step 4 - Evidence Required for Causal Claim:}
\begin{itemize}
    \item Mendelian Randomization showing consistent causal direction
    \item Functional studies (TCF7L2 knockdown impairs insulin secretion)
    \item MODY cases with TCF7L2 mutations (not yet identified)
\end{itemize}

\textbf{Correct Statement:} ``TCF7L2 variants are \textit{associated with} increased T2D risk'' (NOT ``TCF7L2 \textit{causes} T2D'')
\end{tcolorbox}

\subsubsection{Ideal Example C-2: Mendelian Randomization Interpretation}

\begin{tcolorbox}[colback=green!5,colframe=green!40!black,title=C: MR Evidence with Biological Interpretation]
\textbf{Question:} Does Mendelian Randomization evidence support a causal role for PCSK9 in coronary artery disease (CAD)?

\textbf{Ideal Answer:}
\textbf{Yes, MR strongly supports PCSK9 as a causal factor for CAD.}

\textbf{Step 1 - MR Methodology:}
MR uses genetic variants as instrumental variables. PCSK9 loss-of-function variants (e.g., rs11591147) serve as lifelong ``natural experiments'' of reduced PCSK9 activity.

\textbf{Step 2 - MR Evidence:}
\begin{itemize}
    \item MR score: 0.92 (strong causal support)
    \item Consistent across multiple MR methods (IVW, MR-Egger, weighted median)
    \item No evidence of horizontal pleiotropy (Egger intercept p $>$ 0.05)
\end{itemize}

\textbf{Step 3 - Biological Mechanism Explaining Causality:}
\begin{enumerate}
    \item PCSK9 binds to LDL receptors on hepatocytes
    \item PCSK9-LDLr complex is internalized and degraded
    \item Reduced LDL receptor recycling $\rightarrow$ decreased LDL clearance
    \item Elevated circulating LDL-cholesterol
    \item LDL accumulation in arterial walls $\rightarrow$ atherosclerosis $\rightarrow$ CAD
\end{enumerate}

\textbf{Step 4 - Therapeutic Validation (strongest causal support):}
PCSK9 inhibitors (evolocumab, alirocumab) in randomized trials:
\begin{itemize}
    \item Reduce LDL-C by 50-60\%
    \item Reduce cardiovascular events by 15-20\%
    \item This intervention evidence confirms the causal pathway
\end{itemize}

\textbf{Conclusion:} PCSK9 has robust causal evidence linking it to CAD through the LDL-cholesterol pathway, validated by both MR and drug intervention trials.
\end{tcolorbox}

%% ============================================================================
\subsection{Risk-Aware (R) Taxonomy}
%% ============================================================================

\subsubsection{Ideal Example R-1: Risk Factor Assessment}

\begin{tcolorbox}[colback=orange!5,colframe=orange!40!black,title=R: Risk Assessment with Biological Interpretation]
\textbf{Question:} Should APOE be considered a genetic risk factor for Alzheimer's disease? Explain the biological basis.

\textbf{Ideal Answer:}
\textbf{Yes, APOE is the strongest genetic risk factor for late-onset Alzheimer's disease.}

\textbf{Statistical Evidence:}
\begin{itemize}
    \item APOE $\varepsilon$4 allele: OR = 3.2 (heterozygotes), OR = 14.9 (homozygotes)
    \item Genome-wide significance: p $<$ 10$^{-300}$
    \item Risk score: 0.95 (very strong)
    \item Replicated across all populations studied
\end{itemize}

\textbf{Biological Mechanism - Why APOE Increases AD Risk:}

\textbf{1. Amyloid-$\beta$ Clearance:}
\begin{itemize}
    \item APOE is the primary cholesterol transporter in the brain
    \item APOE4 isoform has reduced efficiency in clearing A$\beta$ from brain parenchyma
    \item Results in increased amyloid plaque deposition
\end{itemize}

\textbf{2. Tau Pathology:}
\begin{itemize}
    \item APOE4 enhances tau phosphorylation and tangle formation
    \item Independent of amyloid, APOE4 accelerates tau spread
\end{itemize}

\textbf{3. Neuroinflammation:}
\begin{itemize}
    \item APOE4 promotes pro-inflammatory microglial activation
    \item Impaired synaptic pruning and neuronal support
\end{itemize}

\textbf{4. Blood-Brain Barrier Integrity:}
\begin{itemize}
    \item APOE4 associated with increased BBB permeability
    \item Allows peripheral immune infiltration
\end{itemize}

\textbf{Clinical Implication:}
APOE genotyping has predictive value for AD risk stratification. APOE4 carriers may benefit from earlier intervention and lifestyle modifications. However, APOE4 is neither necessary nor sufficient for AD---it modifies risk within a multifactorial disease context.
\end{tcolorbox}

\subsubsection{Ideal Example R-2: Odds Ratio Interpretation}

\begin{tcolorbox}[colback=orange!5,colframe=orange!40!black,title=R: OR Interpretation with Clinical Context]
\textbf{Question:} What does an odds ratio of 1.45 for GCK variants in Type 2 Diabetes mean clinically?

\textbf{Ideal Answer:}

\textbf{Statistical Interpretation:}
OR = 1.45 means carriers of the GCK risk allele have 45\% higher odds of developing T2D compared to non-carriers.

\textbf{Clinical Context:}
\begin{itemize}
    \item \textbf{Effect Size Classification}: Moderate effect for a common variant (most GWAS variants have OR 1.1-1.3)
    \item \textbf{Population Attributable Risk}: Given allele frequency $\sim$30\%, GCK contributes $\sim$12\% of population T2D risk
    \item \textbf{Individual Risk}: If baseline lifetime T2D risk is 30\%, a GCK risk allele carrier has $\sim$37\% risk
\end{itemize}

\textbf{Biological Explanation for Effect Size:}
GCK encodes glucokinase, the pancreatic glucose sensor. The moderate OR reflects:
\begin{enumerate}
    \item Partial loss of glucose sensing (not complete knockout)
    \item Compensatory mechanisms (other hexokinases, insulin sensitivity)
    \item Gene-environment interaction (effect stronger with obesity/poor diet)
\end{enumerate}

\textbf{Comparison to Other T2D Genes:}
\begin{itemize}
    \item TCF7L2 rs7903146: OR = 1.37 (strongest common variant effect)
    \item GCK: OR = 1.45 (among strongest)
    \item Most T2D GWAS hits: OR = 1.05-1.15
\end{itemize}

\textbf{Important Caveat:}
OR = 1.45 reflects \textit{relative risk increase}, not absolute risk. A 45\% increase from a 1\% baseline risk is trivial (1\% $\rightarrow$ 1.45\%), while from 30\% is meaningful (30\% $\rightarrow$ 37\%).
\end{tcolorbox}

%% ============================================================================
\subsection{Mechanism-Aware (M) Taxonomy}
%% ============================================================================

\subsubsection{Ideal Example M-1: Pathway Analysis}

\begin{tcolorbox}[colback=purple!5,colframe=purple!40!black,title=M: Mechanism with GO/PPI Biological Context]
\textbf{Question:} Based on PPI networks and GO annotations, what is the mechanistic basis for HLA-DRB1's role in Rheumatoid Arthritis?

\textbf{Ideal Answer:}

\textbf{GO Functional Score: 0.85 (Strong pathway relevance)}

\textbf{Relevant GO Terms:}
\begin{itemize}
    \item GO:0002504 - Antigen processing and presentation via MHC class II
    \item GO:0019886 - Antigen processing, exogenous antigen via MHC-II
    \item GO:0050852 - T cell receptor signaling pathway
    \item GO:0002250 - Adaptive immune response
\end{itemize}

\textbf{PPI Network Analysis:}
HLA-DRB1 interacts with:
\begin{itemize}
    \item CD4 (T helper cell co-receptor) - Direct binding
    \item CIITA (MHC class II transactivator) - Transcriptional regulation
    \item CD74 (Invariant chain) - Peptide loading complex
    \item PTPN22 (Another RA risk gene) - T cell activation modulation
\end{itemize}

\textbf{Mechanistic Pathway to Rheumatoid Arthritis:}

\begin{verbatim}
HLA-DRB1*04:01 (shared epitope)
        |
        | [presents citrullinated peptides]
        v
Autoreactive CD4+ T cell activation
        |
        | [provides help to B cells]
        v
Anti-citrullinated protein antibodies (ACPA)
        |
        | [form immune complexes]
        v
Synovial inflammation
        |
        | [recruits macrophages, releases cytokines]
        v
Pannus formation → Cartilage/bone destruction → RA
\end{verbatim}

\textbf{The ``Shared Epitope'' Hypothesis:}
Specific HLA-DRB1 alleles (DRB1*04:01, *04:04, *01:01) share a conserved amino acid sequence (positions 70-74: QKRAA/QRRAA) that:
\begin{enumerate}
    \item Preferentially binds citrullinated self-peptides
    \item Presents these peptides to autoreactive T cells
    \item Breaks self-tolerance, initiating autoimmunity
\end{enumerate}

\textbf{Conclusion:}
HLA-DRB1's mechanistic role in RA is well-established through antigen presentation of citrullinated peptides, supported by PPI interactions with the T cell activation machinery and GO enrichment in adaptive immunity pathways.
\end{tcolorbox}

\subsubsection{Ideal Example M-2: Functional Connection Assessment}

\begin{tcolorbox}[colback=purple!5,colframe=purple!40!black,title=M: Functional Connection with Pathway Explanation]
\textbf{Question:} Is SLC30A8 functionally connected to Type 2 Diabetes based on PPI networks and GO terms?

\textbf{Ideal Answer:}
\textbf{Yes, SLC30A8 has strong functional connection to T2D (GO score: 0.78)}

\textbf{Gene Function:}
SLC30A8 encodes ZnT8, a zinc transporter specifically expressed in pancreatic beta cells.

\textbf{Relevant GO Terms:}
\begin{itemize}
    \item GO:0006882 - Cellular zinc ion homeostasis
    \item GO:0030073 - Insulin secretion
    \item GO:0005385 - Zinc ion transmembrane transporter activity
    \item GO:0005783 - Endoplasmic reticulum (insulin processing)
\end{itemize}

\textbf{Mechanistic Connection to T2D:}
\begin{enumerate}
    \item \textbf{Zinc in Insulin Crystallization:}
    \begin{itemize}
        \item Insulin is stored as zinc-insulin hexamers in secretory granules
        \item ZnT8 transports Zn$^{2+}$ into granules for proper insulin packaging
    \end{itemize}

    \item \textbf{Insulin Secretion:}
    \begin{itemize}
        \item Adequate zinc is required for insulin biosynthesis and processing
        \item ZnT8 dysfunction impairs glucose-stimulated insulin secretion
    \end{itemize}

    \item \textbf{Beta Cell Health:}
    \begin{itemize}
        \item Zinc has antioxidant properties protecting beta cells
        \item ZnT8 autoantibodies are T1D biomarkers (immune relevance)
    \end{itemize}
\end{enumerate}

\textbf{PPI Partners:}
\begin{itemize}
    \item INS (Insulin) - Direct functional link
    \item IAPP (Islet amyloid polypeptide) - Co-secreted with insulin
    \item PCSK1 (Prohormone convertase) - Insulin processing
\end{itemize}

\textbf{Paradoxical Finding:}
Interestingly, SLC30A8 loss-of-function variants are \textit{protective} against T2D (OR = 0.65), suggesting that reduced ZnT8 activity may have compensatory benefits---an active area of drug development research.
\end{tcolorbox}

%% ============================================================================
\subsection{Key Principles for Biological Answers}
%% ============================================================================

\begin{enumerate}
    \item \textbf{Explain the Gene's Function}: What does the gene encode? What is its molecular/cellular role?

    \item \textbf{Connect to Disease Pathophysiology}: How does gene dysfunction lead to disease phenotype?

    \item \textbf{Describe the Pathway}: Provide a step-by-step mechanistic pathway from gene $\rightarrow$ disease.

    \item \textbf{Interpret Statistics Biologically}: Explain why effect sizes are what they are based on biology.

    \item \textbf{Acknowledge Complexity}: Note compensatory mechanisms, gene-environment interactions, and limitations.

    \item \textbf{Clinical Relevance}: Where applicable, mention therapeutic implications.
\end{enumerate}

%% ============================================================================
\subsection{Comparison: Generic vs. Biological Answers}
%% ============================================================================

\begin{table}[h]
\centering
\caption{Generic vs. Biological Answer Quality}
\begin{tabular}{p{6cm}p{6cm}}
\toprule
\textbf{Generic Answer (Poor)} & \textbf{Biological Answer (Good)} \\
\midrule
``Yes. The association is supported by 136 SNPs (20 unique).'' &
``Yes. IL21R encodes the IL-21 receptor, a key regulator of Th2 immune responses. The 136 SNPs (20 independent signals) affect IL-21 signaling, which modulates IgE production and eosinophilic inflammation in asthmatic airways.'' \\
\midrule
``HLA-DRB5 is a protein\_coding gene with known functions.'' &
``HLA-DRB5 encodes an MHC class II molecule that presents antigens to CD4+ T cells. Its role in asthma involves presentation of allergen-derived peptides, activating Th2 responses that drive allergic airway inflammation.'' \\
\midrule
``MR score of 1.00 supports a causal relationship.'' &
``MR score of 1.00 supports causation. Genetically-determined variation in PCSK9 affects LDL receptor degradation, causing lifelong differences in LDL-C levels. This natural experiment mirrors PCSK9 inhibitor trials, confirming the LDL-CAD causal pathway.'' \\
\bottomrule
\end{tabular}
\end{table}

%% ============================================================================
%% END OF BIOLOGICAL EXAMPLES DOCUMENT
%% ============================================================================
