%% BioREASONIC-Bench: Prompts for Generating Expert-Style Answers
%% These prompts produce natural, biologically-informed explanations

\section{Expert Answer Generation Prompts}

This document provides the prompts needed to generate expert-style answers that include proper biological explanations, not mechanical step-by-step templates.

%% ============================================================================
\subsection{System Prompt for Expert Answer Generation}
%% ============================================================================

\begin{tcolorbox}[colback=gray!10,colframe=gray!50!black,title=System Prompt]
\begin{verbatim}
You are an expert biomedical geneticist with deep knowledge of:
- GWAS methodology and interpretation
- Mendelian Randomization and causal inference
- Molecular biology and disease pathophysiology
- Clinical genetics and precision medicine

When answering questions about gene-disease associations:

1. EXPLAIN THE BIOLOGY: Don't just cite numbers. Explain what the gene
   encodes, its molecular function, and how dysfunction leads to disease.

2. USE NATURAL LANGUAGE: Write as an expert explaining to a colleague,
   not as a template being filled in. Avoid "Step 1:", "Step 2:" formats.

3. PROVIDE CONTEXT: Put statistics in perspective. Compare effect sizes
   to typical GWAS findings. Explain what odds ratios mean clinically.

4. CONNECT EVIDENCE TO MECHANISM: Link statistical findings (GWAS, MR)
   to biological pathways that explain WHY the association exists.

5. ACKNOWLEDGE NUANCE: Note limitations, alternative explanations, and
   areas of uncertainty. Use appropriate hedging language.

6. BE OBJECTIVE: State conclusions based on evidence without personal
   opinion framing like "I think" or "My interpretation."
\end{verbatim}
\end{tcolorbox}

%% ============================================================================
\subsection{S (Structure) Taxonomy Prompt}
%% ============================================================================

\begin{tcolorbox}[colback=blue!5,colframe=blue!40!black,title=S-Taxonomy Expert Prompt]
\begin{verbatim}
PROMPT:
You are explaining the genetic architecture of a gene-disease association.

Gene: {gene}
Disease: {disease}
SNP count: {snp_count}
Unique independent signals: {unique_snps}
Evidence level: {evidence_level}

Question: Is the {gene}-{disease} association supported by multiple
independent genetic variants?

Instructions:
1. Answer yes/no with the SNP statistics
2. Explain what "independent signals" means biologically (LD, fine-mapping)
3. Describe what {gene} encodes and its biological function
4. Explain HOW {gene} dysfunction contributes to {disease} pathophysiology
5. Interpret what multiple independent signals suggest about the
   gene's causal role
6. Write in natural expert prose, not numbered steps
\end{verbatim}
\end{tcolorbox}

\textbf{Example Output:}
\begin{quote}
Yes, and the genetic architecture strongly supports IL21R as a true asthma susceptibility gene.

The association is backed by 136 SNPs, of which 20 represent independent signals after accounting for linkage disequilibrium. This distinction matters: a single causal variant can produce many associated SNPs through LD, but 20 independent signals suggest multiple functional variants within or near IL21R affecting asthma risk.

From a biological standpoint, this makes sense. IL21R encodes the receptor for interleukin-21, a cytokine that orchestrates immune responses. In asthma:
- IL-21 signaling drives B cell differentiation and IgE class switching
- It modulates the Th1/Th2 balance (asthma is Th2-driven)
- IL-21 affects regulatory T cells that suppress excessive immune responses

The presence of multiple independent signals suggests several variants—affecting expression, splicing, or protein function—each contribute to asthma susceptibility through this immunological pathway.
\end{quote}

%% ============================================================================
\subsection{C (Causal) Taxonomy Prompt}
%% ============================================================================

\begin{tcolorbox}[colback=green!5,colframe=green!40!black,title=C-Taxonomy Expert Prompt]
\begin{verbatim}
PROMPT:
You are explaining causal inference in genetic epidemiology.

Gene: {gene}
Disease: {disease}
GWAS OR: {or_value}
GWAS p-value: {p_value}
MR score: {mr_score}
Has MR support: {has_mr_support}

Question: Based on GWAS evidence alone, can we conclude that {gene}
causes {disease}?

Instructions:
1. Clearly state NO - GWAS shows association, not causation
2. Explain WHY association ≠ causation (confounding, LD, reverse causation)
3. Describe what {gene} actually does biologically
4. Explain the biological pathway: {gene} → protein → function → {disease}
5. Describe what additional evidence (MR, functional studies) would be
   needed and why
6. If MR evidence exists, explain what it adds to the causal argument
7. Provide the CORRECT language: "associated with" not "causes"
8. Write as a natural explanation, not mechanical steps
\end{verbatim}
\end{tcolorbox}

\textbf{Example Output:}
\begin{quote}
No—and this distinction is crucial for interpreting genetic studies correctly.

GWAS tells us that people carrying TCF7L2 variants (particularly rs7903146) have higher T2D rates. The association is robust: OR of 1.37, replicated across populations. But association is not causation.

Consider the alternatives:

\textbf{Confounding:} Perhaps TCF7L2 variants are more common in populations with dietary patterns that independently increase T2D risk.

\textbf{Linkage disequilibrium:} rs7903146 might tag the true causal variant nearby. TCF7L2 spans over 200kb.

\textbf{Reverse causation:} Less likely for germline variants, but metabolic changes could theoretically affect TCF7L2 regulation.

TCF7L2 is actually well-supported causally. It encodes a transcription factor in Wnt signaling that regulates beta cell proliferation and GLP-1 responses. Functional studies show the risk allele reduces insulin secretion. MR analyses support causation.

But here's the key point: we need that additional evidence beyond GWAS. The GWAS alone establishes association, not causation. Correct statement: "TCF7L2 variants are associated with increased T2D risk."
\end{quote}

%% ============================================================================
\subsection{R (Risk) Taxonomy Prompt}
%% ============================================================================

\begin{tcolorbox}[colback=orange!5,colframe=orange!40!black,title=R-Taxonomy Expert Prompt]
\begin{verbatim}
PROMPT:
You are explaining genetic risk factor assessment.

Gene: {gene}
Disease: {disease}
Risk score: {risk_score}
Evidence level: {evidence_level}
SNP count: {snp_count}
OR: {or_value} (if available)
MR score: {mr_score}

Question: Should {gene} be considered a genetic risk factor for {disease}?

Instructions:
1. State the conclusion (yes/no) with evidence strength
2. Provide the statistical evidence (OR, SNPs, risk score)
3. PUT NUMBERS IN CONTEXT: Compare to typical GWAS effect sizes
4. Explain WHAT {gene} encodes and its molecular function
5. Describe the BIOLOGICAL MECHANISM linking {gene} to {disease}
6. Explain what the effect size means for individual risk
7. Note clinical implications if relevant
8. Write as natural expert explanation
\end{verbatim}
\end{tcolorbox}

\textbf{Example Output:}
\begin{quote}
APOE isn't just a risk factor for Alzheimer's—it's the most important genetic risk factor by a considerable margin.

The numbers are striking. Carrying one APOE ε4 allele roughly triples Alzheimer's risk; two copies increases it 12-fold. No other common variant comes close. For context, most GWAS hits have ORs between 1.1-1.3.

Why such a large effect? APOE is the brain's primary cholesterol transporter, intimately involved in multiple Alzheimer's pathways:

\textbf{Amyloid clearance:} The ε4 isoform is less efficient at clearing Aβ from brain tissue. Aβ accumulates, forms plaques, triggers neurodegeneration.

\textbf{Tau pathology:} APOE4 promotes tau phosphorylation and tangle spread, independent of amyloid.

\textbf{Neuroinflammation:} APOE4 shifts microglia toward pro-inflammatory states.

Clinically, APOE genotyping is used for risk stratification. However, APOE4 is neither necessary nor sufficient for Alzheimer's—many carriers never develop dementia. It's a powerful risk modifier within a multifactorial disease.
\end{quote}

%% ============================================================================
\subsection{M (Mechanism) Taxonomy Prompt}
%% ============================================================================

\begin{tcolorbox}[colback=purple!5,colframe=purple!40!black,title=M-Taxonomy Expert Prompt]
\begin{verbatim}
PROMPT:
You are explaining the biological mechanism connecting a gene to disease.

Gene: {gene}
Disease: {disease}
GO functional score: {go_score}
Has pathway support: {has_pathway_support}
PPI partners: {ppi_partners} (if available)

Question: What is the mechanistic basis for {gene}'s role in {disease}?

Instructions:
1. Start with WHAT the gene encodes (protein name, function)
2. Describe relevant GO terms and what they mean biologically
3. Explain the PPI network context (what proteins it interacts with)
4. MOST IMPORTANT: Describe the step-by-step biological pathway:
   Gene → Protein → Molecular function → Cellular effect →
   Tissue/organ impact → Disease phenotype
5. Connect pathway to disease pathophysiology
6. Note if the mechanism is well-established or hypothetical
7. Mention clinical/therapeutic implications if relevant
8. Write as cohesive biological narrative, not bullet points
\end{verbatim}
\end{tcolorbox}

\textbf{Example Output:}
\begin{quote}
HLA-DRB1 is central to rheumatoid arthritis pathogenesis, and we understand the mechanism in considerable detail—a rarity in complex disease genetics.

HLA-DRB1 encodes part of the MHC class II complex, presenting peptide fragments to CD4+ T cells. RA-associated alleles (*04:01, *04:04, *01:01) share a common amino acid sequence at positions 70-74: the "shared epitope."

Here's how this drives RA:

In RA, the immune system attacks joints. The current model involves citrullination—arginine residues converted to citrulline by PAD enzymes during inflammation.

The shared epitope preferentially binds citrullinated peptides. When citrullinated self-proteins (fibrinogen, vimentin, collagen) are presented by HLA-DR molecules carrying the shared epitope, they activate T cells that escaped tolerance.

These autoreactive T cells then:
1. Help B cells produce anti-citrullinated protein antibodies (ACPAs)
2. ACPAs form immune complexes in the synovium
3. Immune complexes activate complement, recruit macrophages
4. Chronic inflammation leads to pannus formation—invasive tissue destroying cartilage and bone

The PPI network reflects this: HLA-DRB1 interacts with CD4 (T cell co-receptor), CIITA (MHC II master regulator), and CD74 (peptide loading). Another RA risk gene, PTPN22, also modulates T cell activation—the genetic architecture converges on the same pathway.
\end{quote}

%% ============================================================================
\subsection{Key Differences: Current vs. Expert Prompts}
%% ============================================================================

\begin{table}[h]
\centering
\caption{Current Mechanical Prompts vs. Expert Prompts}
\begin{tabular}{p{6cm}p{6cm}}
\toprule
\textbf{Current Prompt (Mechanical)} & \textbf{Expert Prompt (Biological)} \\
\midrule
``Generate a step-by-step answer'' &
``Write as an expert explaining to a colleague'' \\
\midrule
``Step 1: Count variants. Step 2: Assess independence...'' &
``Explain what independent signals mean biologically, then describe the gene's function and pathway to disease'' \\
\midrule
``{gene} is a protein\_coding gene with known functions'' &
``Describe what {gene} encodes, its molecular function, and how dysfunction leads to {disease}'' \\
\midrule
``MR score = {mr\_score}'' &
``Explain what MR methodology does, interpret the score, and connect to biological mechanism'' \\
\midrule
No biological context required &
MUST include: Gene → Protein → Function → Disease pathway \\
\bottomrule
\end{tabular}
\end{table}

%% ============================================================================
\subsection{Required Knowledge Base for Expert Answers}
%% ============================================================================

To generate expert answers, the system needs access to:

\begin{enumerate}
    \item \textbf{Gene Function Database}: What each gene encodes, molecular function
    \item \textbf{Pathway Information}: GO terms, KEGG pathways, Reactome
    \item \textbf{Disease Pathophysiology}: How diseases develop at molecular/cellular level
    \item \textbf{PPI Networks}: Protein interaction partners
    \item \textbf{Literature Context}: Key findings, drug targets, clinical relevance
\end{enumerate}

Without this biological knowledge, answers remain mechanical number-reporting rather than expert explanations.

%% ============================================================================
%% END OF EXPERT GENERATION PROMPTS
%% ============================================================================
