%% BioREASONIC-Bench: Prompts and Question Examples
%% Comprehensive documentation of generation prompts and best examples for each taxonomy

\section{Question Generation Prompts and Examples}

This document provides the complete set of generation prompts and representative examples for each taxonomy in BioREASONIC-Bench. Each taxonomy is presented with: (1) the generation prompt template, (2) answer templates, and (3) high-quality generated examples across different answer formats.

%% ============================================================================
\subsection{Structure-Aware (S) Taxonomy}
%% ============================================================================

\subsubsection{Generation Prompt Templates}

\paragraph{S-SNP-COUNT Template:}
Evaluates understanding of genetic variant support for gene-disease associations.

\begin{verbatim}
PROMPT: "Evaluate the robustness of the {gene}-{disease} association
        based on supporting genetic variants."

ANSWER TEMPLATE (Reasoning):
Step 1: Count supporting variants - {snp_count} total SNPs,
        {unique_snps} unique.
Step 2: Assess independence - Multiple independent signals suggest
        true association.
Step 3: Consider LD structure - Considering LD structure, {unique_snps}
        unique variants from {snp_count} total SNPs suggest multiple
        causal variants.
Step 4: Evaluate effect sizes - Effect sizes are typical for complex
        trait associations.
Step 5: Check replication - {replication_status}.
Conclusion: The {gene}-{disease} association is {robustness} based on
            {snp_count} supporting SNPs with {unique_snps} unique variants.
\end{verbatim}

\subsubsection{Generated Examples}

\paragraph{Example S-1: Yes/No Format}
\begin{tcolorbox}[colback=blue!5,colframe=blue!40!black,title=S-SNP-COUNT (Yes/No)]
\textbf{Question:} Is the IL21R-Asthma association supported by multiple independent genetic variants?

\textbf{Answer:} Yes. The association is supported by 136 SNPs (20 unique).

\textbf{Evidence:}
\begin{itemize}
    \item Gene: IL21R (ENSG00000103522)
    \item Disease: Asthma
    \item SNP Count: 136 total, 20 unique
    \item Evidence Level: Very Strong (MR-validated)
    \item Risk Score: 0.71
\end{itemize}
\end{tcolorbox}

\paragraph{Example S-2: Reasoning Format}
\begin{tcolorbox}[colback=blue!5,colframe=blue!40!black,title=S-SNP-COUNT (Reasoning)]
\textbf{Question:} Evaluate the robustness of the DHRS7C-Diabetes Mellitus, Type 2 association based on supporting genetic variants.

\textbf{Answer:}
\begin{verbatim}
Step 1: Count supporting variants - 30 total SNPs, 30 unique.
Step 2: Assess independence - Multiple independent signals suggest
        true association.
Step 3: Consider LD structure - Considering LD structure, 30 unique
        variants from 30 total SNPs suggest multiple causal variants.
Step 4: Evaluate effect sizes - Effect sizes are typical for complex
        trait associations.
Step 5: Check replication - Partially replicated.
Conclusion: The DHRS7C-Diabetes Mellitus, Type 2 association is
            moderately robust based on 30 supporting SNPs with
            30 unique variants.
\end{verbatim}
\end{tcolorbox}

%% ============================================================================
\subsection{Causal-Aware (C) Taxonomy}
%% ============================================================================

\subsubsection{Generation Prompt Templates}

\paragraph{C-CAUSAL-VS-ASSOC Template:}
The \textbf{core} template for distinguishing association from causation.

\begin{verbatim}
PROMPT: "Is the relationship between {gene} and {disease} causal or
        associative based on GWAS evidence?"

ANSWER TEMPLATE (CoT):
The relationship is ASSOCIATIVE, not causal.

**Step 1 - Evidence Type:**
GWAS provides statistical association (OR={or_value}, p={p_value}).

**Step 2 - Causal Limitation:**
GWAS identifies correlation but CANNOT prove causation due to:
- Potential confounding variables
- Possible reverse causation
- Linkage disequilibrium with true causal variant

**Step 3 - Required Evidence for Causality:**
- Mendelian Randomization (MR) analysis
- Functional studies showing biological mechanism
- Intervention trials

**Step 4 - Conclusion:**
Based on GWAS alone, we can only say {gene} is ASSOCIATED with
{disease}, not that it CAUSES {disease}.
\end{verbatim}

\paragraph{C-MR-EVIDENCE Template:}
Evaluates interpretation of Mendelian Randomization evidence.

\begin{verbatim}
PROMPT: "Based on Mendelian Randomization analysis, can we infer that
        {gene} causally affects {disease}? Explain your reasoning."

ANSWER TEMPLATE (CoT):
Step 1: Understand MR methodology - MR uses genetic variants as
        natural experiments to test causation.
Step 2: Examine MR score - {gene} has an MR score of {mr_score}.
Step 3: Interpret the score - [Strong/Moderate/Insufficient] MR support
        provides causal evidence through natural genetic randomization.
Step 4: Consider MR assumptions - Relevance, independence,
        exclusion restriction.
Step 5: Assess potential violations - [Assessment].
Conclusion: [supports/does not support] a causal relationship.
\end{verbatim}

\paragraph{C-CAUSAL-STRENGTH Template:}
Distinguishes association from causation with evidence evaluation.

\begin{verbatim}
PROMPT: "Evaluate whether the evidence supports a causal role for {gene}
        in {disease}. Distinguish between association and causation."

ANSWER TEMPLATE (CoT):
Step 1: Define association vs causation - Association shows correlation;
        causation requires evidence the gene directly affects disease risk.
Step 2: Review associative evidence - {snp_count} GWAS SNPs,
        risk score {risk_score}.
Step 3: Examine causal evidence - Causal confidence: {causal_score},
        MR score: {mr_score}.
Step 4: Apply causal criteria - [Assessment of criteria].
Step 5: Consider confounding - MR design reduces confounding.
Conclusion: Evidence [supports/does not support] a causal role.
\end{verbatim}

\subsubsection{Generated Examples}

\paragraph{Example C-1: Yes/No Format}
\begin{tcolorbox}[colback=green!5,colframe=green!40!black,title=C-MR-EVIDENCE (Yes/No)]
\textbf{Question:} Does HLA-DRB5 have Mendelian Randomization evidence supporting a causal role in Anti-Asthmatic Agents?

\textbf{Answer:} Yes. MR score of 1.00 supports a causal relationship.

\textbf{Evidence:}
\begin{itemize}
    \item Gene: HLA-DRB5
    \item MR Score: 1.00 (Strong causal support)
    \item Evidence Level: Very Strong
    \item Has MR Support: True
\end{itemize}
\end{tcolorbox}

\paragraph{Example C-2: Multiple Choice Format}
\begin{tcolorbox}[colback=green!5,colframe=green!40!black,title=C-MR-EVIDENCE (MCQ)]
\textbf{Question:} What does the Mendelian Randomization evidence suggest about HLA-DRB5's relationship with Anti-Asthmatic Agents?

\textbf{Options:}
\begin{itemize}
    \item[A)] Strong causal evidence (MR supports causation) \checkmark
    \item[B)] Moderate causal evidence (MR suggestive)
    \item[C)] Association only (no MR support for causation)
    \item[D)] No evidence of relationship
\end{itemize}

\textbf{Correct Answer:} A
\end{tcolorbox}

\paragraph{Example C-3: Reasoning Format (Best Example)}
\begin{tcolorbox}[colback=green!5,colframe=green!40!black,title=C-CAUSAL-STRENGTH (Reasoning)]
\textbf{Question:} Evaluate whether the evidence supports a causal role for DHRS7C in Diabetes Mellitus, Type 2. Distinguish between association and causation.

\textbf{Answer:}
\begin{verbatim}
Step 1: Define association vs causation - Association shows correlation;
        causation requires evidence the gene directly affects disease risk.

Step 2: Review associative evidence - 30 GWAS SNPs, risk score 0.54.

Step 3: Examine causal evidence - Causal confidence: 0.02,
        MR score: 1.00.

Step 4: Apply causal criteria - Limited support for causal criteria.

Step 5: Consider confounding - MR design reduces confounding.

Conclusion: Evidence supports a causal role. The evidence supports
            causation.
\end{verbatim}

\textbf{Ground Truth Evidence:}
\begin{itemize}
    \item GWAS Support: 30 SNPs (moderate)
    \item MR Score: 1.00 (strong)
    \item Causal Confidence: 0.02
    \item Evidence Level: Moderate
\end{itemize}
\end{tcolorbox}

%% ============================================================================
\subsection{Risk-Aware (R) Taxonomy}
%% ============================================================================

\subsubsection{Generation Prompt Templates}

\paragraph{R-RISK-FACTOR Template:}
\begin{verbatim}
PROMPT: "Based on the available genetic evidence, should {gene} be
        considered a risk factor for {disease}? Explain your reasoning."

ANSWER TEMPLATE (CoT):
Step 1: Examine the GWAS evidence - {gene} has {snp_count} associated
        SNPs with {disease}.
Step 2: Assess evidence strength - The risk weight score is {risk_score},
        classified as {evidence_level}.
Step 3: Evaluate supporting data - {evidence_description}.
Step 4: Consider biological plausibility - {gene} is a {gene_type} gene
        with known functions.
Conclusion: [Yes/No], {gene} [should/should not] be considered a
            [significant] risk factor for {disease}.
\end{verbatim}

\paragraph{R-RISK-LEVEL Template:}
\begin{verbatim}
PROMPT: "Evaluate the strength of evidence for {gene} as a risk factor
        for {disease}. What does the evidence tell us?"

ANSWER TEMPLATE (CoT):
Step 1: Check the risk weight score - {risk_score} indicates
        {evidence_level} evidence.
Step 2: Examine GWAS support - {snp_count} SNPs support this association.
Step 3: Consider causal confidence - The causal confidence score
        is {causal_score}.
Step 4: Assess MR evidence - MR score is {mr_score}, indicating
        {mr_interpretation}.
Conclusion: The overall evidence is {evidence_level}.
\end{verbatim}

\subsubsection{Generated Examples}

\paragraph{Example R-1: Yes/No Format}
\begin{tcolorbox}[colback=orange!5,colframe=orange!40!black,title=R-RISK-FACTOR (Yes/No)]
\textbf{Question:} Is IL21R a genetic risk factor for Asthma?

\textbf{Answer:} Yes. IL21R is a risk factor for Asthma with very strong (MR-validated) evidence (risk score: 0.71).

\textbf{Evidence:}
\begin{itemize}
    \item Gene: IL21R
    \item Disease: Asthma
    \item Risk Weight Score: 0.71
    \item Evidence Level: Very Strong (MR-validated)
    \item SNP Count: 136
\end{itemize}
\end{tcolorbox}

\paragraph{Example R-2: Reasoning Format (Best Example)}
\begin{tcolorbox}[colback=orange!5,colframe=orange!40!black,title=R-RISK-FACTOR (Reasoning)]
\textbf{Question:} Based on the available genetic evidence, should DHRS7C be considered a risk factor for Diabetes Mellitus, Type 2? Explain your reasoning.

\textbf{Answer:}
\begin{verbatim}
Step 1: Examine the GWAS evidence - DHRS7C has 30 associated SNPs
        with Diabetes Mellitus, Type 2.

Step 2: Assess evidence strength - The risk weight score is 0.54,
        classified as moderate.

Step 3: Evaluate supporting data - moderate GWAS support (30 SNPs);
        strong Mendelian Randomization evidence.

Step 4: Consider biological plausibility - DHRS7C is a protein_coding
        gene with known functions.

Conclusion: DHRS7C may be a risk factor for Diabetes Mellitus, Type 2,
            but more evidence is needed.
\end{verbatim}
\end{tcolorbox}

%% ============================================================================
\subsection{Mechanism-Aware (M) Taxonomy}
%% ============================================================================

\subsubsection{Generation Prompt Templates}

\paragraph{M-PATHWAY Template (Biological Mechanism):}
\begin{verbatim}
PROMPT: "Based on protein-protein interaction networks and GO annotations,
        what is the mechanistic basis for {gene}'s role in {disease}?"

ANSWER TEMPLATE (CoT):
Step 1: Examine PPI network - {gene} participates in disease-relevant
        protein interaction networks (GO score: {go_score}).
Step 2: Analyze GO enrichment - Key GO terms include biological process,
        cellular component, and molecular function terms relevant
        to disease.
Step 3: Assess functional relevance - GO functional score: {go_score}.
Step 4: Connect to disease mechanisms - The gene's PPI network shows
        [strong/moderate/weak] overlap with disease pathways.
Step 5: Consider pathway crosstalk - Cross-pathway interactions may
        contribute to disease phenotype.
Conclusion: The pathway analysis [supports/does not support] a
            mechanistic role for {gene} in {disease}.
\end{verbatim}

\paragraph{M-MECHANISM Template:}
\begin{verbatim}
PROMPT: "Does the PPI network and GO annotation of {gene} support its
        role in {disease} pathogenesis?"

ANSWER TEMPLATE (Yes/No):
Yes/No. GO functional score of {go_score} [supports/does not support]
mechanistic involvement.
\end{verbatim}

\subsubsection{Generated Examples}

\paragraph{Example M-1: Yes/No Format}
\begin{tcolorbox}[colback=purple!5,colframe=purple!40!black,title=M-PATHWAY (Yes/No)]
\textbf{Question:} Is HLA-DRB5 functionally connected to Anti-Asthmatic Agents based on PPI networks and GO term enrichment?

\textbf{Answer:} Yes. PPI/GO functional score of 0.60 indicates functional connection.

\textbf{Evidence:}
\begin{itemize}
    \item Gene: HLA-DRB5
    \item GO Functional Score: 0.60
    \item Has Pathway Support: True
    \item Evidence Level: Very Strong
\end{itemize}
\end{tcolorbox}

\paragraph{Example M-2: Reasoning Format (Best Example)}
\begin{tcolorbox}[colback=purple!5,colframe=purple!40!black,title=M-PATHWAY (Reasoning)]
\textbf{Question:} Based on protein-protein interaction networks and GO annotations, what is the mechanistic basis for HLA-DRB5's role in Anti-Asthmatic Agents?

\textbf{Answer:}
\begin{verbatim}
Step 1: Examine PPI network - HLA-DRB5 participates in disease-relevant
        protein interaction networks (GO score: 0.60).

Step 2: Analyze GO enrichment - Key GO terms include biological process,
        cellular component, and molecular function terms relevant
        to disease.

Step 3: Assess functional relevance - GO functional score: 0.60.

Step 4: Connect to disease mechanisms - The gene's PPI network shows
        strong overlap with disease pathways.

Step 5: Consider pathway crosstalk - Cross-pathway interactions may
        contribute to disease phenotype.

Conclusion: The pathway analysis supports a mechanistic role for
            HLA-DRB5 in Anti-Asthmatic Agents.
\end{verbatim}
\end{tcolorbox}

%% ============================================================================
\subsection{Summary of Answer Templates by Format}
%% ============================================================================

\begin{table}[h]
\centering
\caption{Answer Format Templates Across Taxonomies}
\begin{tabular}{p{2cm}p{3cm}p{8cm}}
\toprule
\textbf{Format} & \textbf{Structure} & \textbf{Example Pattern} \\
\midrule
Yes/No & Direct answer + evidence & ``Yes. [Evidence statement with scores].'' \\
\midrule
MCQ & 4 options (A-D) & Options range from strongest to weakest evidence interpretation \\
\midrule
Short Answer & 1-2 sentences & Concise statement with key metrics \\
\midrule
Long Answer & 3-5 sentences & Detailed explanation with multiple evidence types \\
\midrule
Reasoning (CoT) & 4-5 steps + conclusion & Step-by-step reasoning with explicit evaluation \\
\bottomrule
\end{tabular}
\end{table}

%% ============================================================================
\subsection{Key Generation Principles}
%% ============================================================================

\begin{enumerate}
    \item \textbf{Causal Language Discipline}: C taxonomy questions enforce strict distinction between ``associated with'' (GWAS) and ``causes'' (MR-validated).

    \item \textbf{Evidence Grounding}: All answers include specific scores (MR score, risk score, GO score) from the knowledge graph.

    \item \textbf{Step-by-Step Reasoning}: Reasoning format answers follow a structured 4-5 step CoT template.

    \item \textbf{Uncertainty Quantification}: Conclusions reflect evidence strength (``very strong'', ``moderate'', ``weak'').

    \item \textbf{Reproducibility}: Each question links to source data with gene IDs, disease IDs, and evidence descriptions.
\end{enumerate}

%% ============================================================================
%% END OF PROMPTS AND EXAMPLES DOCUMENT
%% ============================================================================
